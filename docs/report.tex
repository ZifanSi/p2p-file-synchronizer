\documentclass[12pt]{article}
\usepackage[utf8]{inputenc}
\usepackage{geometry}
\geometry{letterpaper, margin=1in}
\usepackage{graphicx}
\usepackage{booktabs}
\usepackage{listings}
\usepackage{xcolor}
\usepackage{hyperref}

% Define colors for code snippets
\definecolor{codegreen}{rgb}{0,0.6,0}
\definecolor{codegray}{rgb}{0.5,0.5,0.5}
\definecolor{backcolour}{rgb}{0.95,0.95,0.92}

\lstset{
    backgroundcolor=\color{backcolour},
    commentstyle=\color{codegreen},
    keywordstyle=\color{magenta},
    numberstyle=\tiny\color{codegray},
    basicstyle=\ttfamily\footnotesize,
    breaklines=true,
    showspaces=false,
    showstringspaces=false,
}

\title{Assignment 2: Peer-to-Peer File Synchronizer Report}
\author{Student Name: \underline{\hspace{4cm}} \\ Student ID: \underline{\hspace{4cm}}}
\date{\today}

\begin{document}

\maketitle

\section{System Overview}
This report details the implementation of a P2P file sharing application designed to synchronize files between end hosts acting as both servers and clients[cite: 7]. The system consists of a centralized tracker for peer discovery and a direct peer-to-peer protocol for binary file transfers[cite: 34, 35].

\section{Implementation of Core Functions}
The following functions were implemented as per the grading rubrics[cite: 166, 167, 168]:

\begin{itemize}
    \item \textbf{\texttt{get\_file\_info()}}: Retrieves filenames and last modified times ($mtime$) as integers (seconds since epoch) from the working directory[cite: 68, 166].
    \item \textbf{\texttt{get\_next\_available\_port()}}: Dynamically binds to an open TCP port for the peer's file-serving listener[cite: 167].
    \item \textbf{Initializer}: Configures the persistent TCP connection to the tracker and initializes the dual-thread model (Peer-to-Tracker and Peer-to-Peer)[cite: 50, 168].
\end{itemize}

\section{Protocol Adherence}
\subsection{Peer-to-Tracker Interaction}
The synchronizer establishes one persistent connection to the tracker[cite: 50]. 
\begin{itemize}
    \item \textbf{Initial Message}: Sent upon startup with the local file list[cite: 57].
    \item \textbf{Keepalive}: Sent every 5 seconds to maintain "live" status[cite: 70].
    \item \textbf{JSON Framing}: All messages are UTF-8 JSON objects terminated by a newline ($\setminus$n)[cite: 54].
\end{itemize}

\subsection{Peer-to-Peer File Transfer}
File requests are handled over separate TCP connections[cite: 104].
\begin{itemize}
    \item \textbf{Binary Mode}: Files are opened using \texttt{"rb"} and \texttt{"wb"} modes to ensure data integrity[cite: 116, 120].
    \item \textbf{Header Specification}: The serving peer sends a \texttt{Content-Length: <size>$\setminus$n} header before transferring raw bytes[cite: 114].
    \item \textbf{Storage and Cleanup}: Partial files from failed or timed-out transfers are explicitly discarded[cite: 125, 128].
\end{itemize}

\section{Test Cases and Results}
As required by the rubric, the following test cases were conducted:

\begin{table}[h!]
\centering
\begin{tabular}{@{}p{3.5cm}p{5.5cm}p{4cm}@{}}
\toprule
\textbf{Test Case} & \textbf{Objective} & \textbf{Outcome} \\ \midrule
Peer Discovery & Verify tracker registers Peer 1 with \texttt{fileA.txt}. & Success \\
File Retrieval & Peer 2 downloads \texttt{fileA.txt} from Peer 1[cite: 172]. & Success \\
Update Logic & Peer 2 overwrites local file with a newer $mtime$ version[cite: 173]. & Success \\
Error Handling & Verify partial file deletion on connection timeout[cite: 174]. & Success \\ \bottomrule
\end{tabular}
\caption{Summary of validation tests performed during development.}
\end{table}

\section{Screenshots}
\begin{figure}[h]
    \centering
    % \includegraphics[width=0.7\textwidth]{tracker_log.png}
    \caption{Tracker log showing registration of Peer 1, 2, and 3[cite: 14].}
\end{figure}

\end{document}